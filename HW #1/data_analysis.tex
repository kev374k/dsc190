% Options for packages loaded elsewhere
\PassOptionsToPackage{unicode}{hyperref}
\PassOptionsToPackage{hyphens}{url}
%
\documentclass[
]{article}
\usepackage{amsmath,amssymb}
\usepackage{iftex}
\ifPDFTeX
  \usepackage[T1]{fontenc}
  \usepackage[utf8]{inputenc}
  \usepackage{textcomp} % provide euro and other symbols
\else % if luatex or xetex
  \usepackage{unicode-math} % this also loads fontspec
  \defaultfontfeatures{Scale=MatchLowercase}
  \defaultfontfeatures[\rmfamily]{Ligatures=TeX,Scale=1}
\fi
\usepackage{lmodern}
\ifPDFTeX\else
  % xetex/luatex font selection
\fi
% Use upquote if available, for straight quotes in verbatim environments
\IfFileExists{upquote.sty}{\usepackage{upquote}}{}
\IfFileExists{microtype.sty}{% use microtype if available
  \usepackage[]{microtype}
  \UseMicrotypeSet[protrusion]{basicmath} % disable protrusion for tt fonts
}{}
\makeatletter
\@ifundefined{KOMAClassName}{% if non-KOMA class
  \IfFileExists{parskip.sty}{%
    \usepackage{parskip}
  }{% else
    \setlength{\parindent}{0pt}
    \setlength{\parskip}{6pt plus 2pt minus 1pt}}
}{% if KOMA class
  \KOMAoptions{parskip=half}}
\makeatother
\usepackage{xcolor}
\usepackage[margin=1in]{geometry}
\usepackage{color}
\usepackage{fancyvrb}
\newcommand{\VerbBar}{|}
\newcommand{\VERB}{\Verb[commandchars=\\\{\}]}
\DefineVerbatimEnvironment{Highlighting}{Verbatim}{commandchars=\\\{\}}
% Add ',fontsize=\small' for more characters per line
\usepackage{framed}
\definecolor{shadecolor}{RGB}{248,248,248}
\newenvironment{Shaded}{\begin{snugshade}}{\end{snugshade}}
\newcommand{\AlertTok}[1]{\textcolor[rgb]{0.94,0.16,0.16}{#1}}
\newcommand{\AnnotationTok}[1]{\textcolor[rgb]{0.56,0.35,0.01}{\textbf{\textit{#1}}}}
\newcommand{\AttributeTok}[1]{\textcolor[rgb]{0.13,0.29,0.53}{#1}}
\newcommand{\BaseNTok}[1]{\textcolor[rgb]{0.00,0.00,0.81}{#1}}
\newcommand{\BuiltInTok}[1]{#1}
\newcommand{\CharTok}[1]{\textcolor[rgb]{0.31,0.60,0.02}{#1}}
\newcommand{\CommentTok}[1]{\textcolor[rgb]{0.56,0.35,0.01}{\textit{#1}}}
\newcommand{\CommentVarTok}[1]{\textcolor[rgb]{0.56,0.35,0.01}{\textbf{\textit{#1}}}}
\newcommand{\ConstantTok}[1]{\textcolor[rgb]{0.56,0.35,0.01}{#1}}
\newcommand{\ControlFlowTok}[1]{\textcolor[rgb]{0.13,0.29,0.53}{\textbf{#1}}}
\newcommand{\DataTypeTok}[1]{\textcolor[rgb]{0.13,0.29,0.53}{#1}}
\newcommand{\DecValTok}[1]{\textcolor[rgb]{0.00,0.00,0.81}{#1}}
\newcommand{\DocumentationTok}[1]{\textcolor[rgb]{0.56,0.35,0.01}{\textbf{\textit{#1}}}}
\newcommand{\ErrorTok}[1]{\textcolor[rgb]{0.64,0.00,0.00}{\textbf{#1}}}
\newcommand{\ExtensionTok}[1]{#1}
\newcommand{\FloatTok}[1]{\textcolor[rgb]{0.00,0.00,0.81}{#1}}
\newcommand{\FunctionTok}[1]{\textcolor[rgb]{0.13,0.29,0.53}{\textbf{#1}}}
\newcommand{\ImportTok}[1]{#1}
\newcommand{\InformationTok}[1]{\textcolor[rgb]{0.56,0.35,0.01}{\textbf{\textit{#1}}}}
\newcommand{\KeywordTok}[1]{\textcolor[rgb]{0.13,0.29,0.53}{\textbf{#1}}}
\newcommand{\NormalTok}[1]{#1}
\newcommand{\OperatorTok}[1]{\textcolor[rgb]{0.81,0.36,0.00}{\textbf{#1}}}
\newcommand{\OtherTok}[1]{\textcolor[rgb]{0.56,0.35,0.01}{#1}}
\newcommand{\PreprocessorTok}[1]{\textcolor[rgb]{0.56,0.35,0.01}{\textit{#1}}}
\newcommand{\RegionMarkerTok}[1]{#1}
\newcommand{\SpecialCharTok}[1]{\textcolor[rgb]{0.81,0.36,0.00}{\textbf{#1}}}
\newcommand{\SpecialStringTok}[1]{\textcolor[rgb]{0.31,0.60,0.02}{#1}}
\newcommand{\StringTok}[1]{\textcolor[rgb]{0.31,0.60,0.02}{#1}}
\newcommand{\VariableTok}[1]{\textcolor[rgb]{0.00,0.00,0.00}{#1}}
\newcommand{\VerbatimStringTok}[1]{\textcolor[rgb]{0.31,0.60,0.02}{#1}}
\newcommand{\WarningTok}[1]{\textcolor[rgb]{0.56,0.35,0.01}{\textbf{\textit{#1}}}}
\usepackage{graphicx}
\makeatletter
\newsavebox\pandoc@box
\newcommand*\pandocbounded[1]{% scales image to fit in text height/width
  \sbox\pandoc@box{#1}%
  \Gscale@div\@tempa{\textheight}{\dimexpr\ht\pandoc@box+\dp\pandoc@box\relax}%
  \Gscale@div\@tempb{\linewidth}{\wd\pandoc@box}%
  \ifdim\@tempb\p@<\@tempa\p@\let\@tempa\@tempb\fi% select the smaller of both
  \ifdim\@tempa\p@<\p@\scalebox{\@tempa}{\usebox\pandoc@box}%
  \else\usebox{\pandoc@box}%
  \fi%
}
% Set default figure placement to htbp
\def\fps@figure{htbp}
\makeatother
\setlength{\emergencystretch}{3em} % prevent overfull lines
\providecommand{\tightlist}{%
  \setlength{\itemsep}{0pt}\setlength{\parskip}{0pt}}
\setcounter{secnumdepth}{-\maxdimen} % remove section numbering
\usepackage{bookmark}
\IfFileExists{xurl.sty}{\usepackage{xurl}}{} % add URL line breaks if available
\urlstyle{same}
\hypersetup{
  pdftitle={Impact of Maternal Smoking on Infant Birthweight},
  pdfauthor={Author 1 and Author 2},
  hidelinks,
  pdfcreator={LaTeX via pandoc}}

\title{Impact of Maternal Smoking on Infant Birthweight}
\author{Author 1 and Author 2}
\date{2024-10-08}

\begin{document}
\maketitle

\section{Header}\label{header}

\subsection{Author Contributions}\label{author-contributions}

Brief description of the respective contribution of each team member.

Author 1: Worked on questions \#1, \#3, and \#5, and created the data
analysis template for the homework

Author 2:

\subsection{Use of GPT}\label{use-of-gpt}

ChatGPT was used as a substitute for documentation for R. Since we were
unfamiliar with R, we asked ChatGPT how to use R in certain methods in
order to find and filter out conditions in the dataset. We additionally
used GPT to analyze reasoning and to confirm what we thought was correct
about the dataset, as well as to identify extra questions that could be
answered for our advanced analysis.

\section{Introduction}\label{introduction}

The data provided is a Child Health and Development Studies dataset,
which consisted of all pregnancies that occurred from 1960-1967 among
women with the \textbf{Kaiser Health Plan} in Oakland, CA. Some
important things to note are that all 1236 babies in the dataset are
boys, there are no twins, and all lived at least 28 days. It's important
to keep in mind that this is not classified as a simple random sample of
all pregnancies, because the conditions just posed cannot be proven to
be a totally random sample of all babies born to mothers. However, we
are still studying this data beause it still should be a decent
representation of differences in weight between babies born to mothers
who smoked during pregnancy and those who didn't, even if it is not
totally representative of all babies.

\subsection{Main Research Questions}\label{main-research-questions}

\begin{enumerate}
\def\labelenumi{\arabic{enumi}.}
\tightlist
\item
  What are the numerical distributions of the birth weight for babies
  born to women who smoked versus those who didn't smoke?
\item
  Is there a significant difference in these two distributions? If so,
  what type of conclusion can be reached?
\item
  What percentage of babies born between these two groups (non-smoking
  mothers and smoking mothers) are considered low-birth-weight babies?
  Is there a difference?
\item
  How does the reliability of the three types of comparisons -
  numerical, graphical, and incidence - change based on our data, and
  which was the best?
\end{enumerate}

\subsection{Outline}\label{outline}

The remainder of the report will go through a basic analysis of the
data, including our cleaning methods, basic analysis on various
variables in the study, and more. Additionally, we will analysis the
questions posed above, along with the conclusions that we came up with
in our data.

\section{Basic Analysis}\label{basic-analysis}

\subsection{Data Processing and
Summaries}\label{data-processing-and-summaries}

\subsubsection{Methods}\label{methods}

\begin{Shaded}
\begin{Highlighting}[]
\NormalTok{data }\OtherTok{\textless{}{-}} \FunctionTok{read.table}\NormalTok{(}\StringTok{"babies.txt"}\NormalTok{, }\AttributeTok{header =} \ConstantTok{TRUE}\NormalTok{)}
\FunctionTok{head}\NormalTok{(data, }\DecValTok{5}\NormalTok{)}
\end{Highlighting}
\end{Shaded}

\begin{verbatim}
##   bwt gestation parity age height weight smoke
## 1 120       284      0  27     62    100     0
## 2 113       282      0  33     64    135     0
## 3 128       279      0  28     64    115     1
## 4 123       999      0  36     69    190     0
## 5 108       282      0  23     67    125     1
\end{verbatim}

\begin{Shaded}
\begin{Highlighting}[]
\FunctionTok{summary}\NormalTok{(data)}
\end{Highlighting}
\end{Shaded}

\begin{verbatim}
##       bwt          gestation         parity            age       
##  Min.   : 55.0   Min.   :148.0   Min.   :0.0000   Min.   :15.00  
##  1st Qu.:108.8   1st Qu.:272.0   1st Qu.:0.0000   1st Qu.:23.00  
##  Median :120.0   Median :280.0   Median :0.0000   Median :26.00  
##  Mean   :119.6   Mean   :286.9   Mean   :0.2549   Mean   :27.37  
##  3rd Qu.:131.0   3rd Qu.:288.0   3rd Qu.:1.0000   3rd Qu.:31.00  
##  Max.   :176.0   Max.   :999.0   Max.   :1.0000   Max.   :99.00  
##      height          weight        smoke       
##  Min.   :53.00   Min.   : 87   Min.   :0.0000  
##  1st Qu.:62.00   1st Qu.:115   1st Qu.:0.0000  
##  Median :64.00   Median :126   Median :0.0000  
##  Mean   :64.67   Mean   :154   Mean   :0.4644  
##  3rd Qu.:66.00   3rd Qu.:140   3rd Qu.:1.0000  
##  Max.   :99.00   Max.   :999   Max.   :9.0000
\end{verbatim}

\begin{Shaded}
\begin{Highlighting}[]
\NormalTok{bwt\_description }\OtherTok{\textless{}{-}} \FunctionTok{summary}\NormalTok{(data}\SpecialCharTok{$}\NormalTok{bwt)}
\FunctionTok{print}\NormalTok{(bwt\_description)}
\end{Highlighting}
\end{Shaded}

\begin{verbatim}
##    Min. 1st Qu.  Median    Mean 3rd Qu.    Max. 
##    55.0   108.8   120.0   119.6   131.0   176.0
\end{verbatim}

\begin{Shaded}
\begin{Highlighting}[]
\NormalTok{gestation\_description }\OtherTok{\textless{}{-}} \FunctionTok{summary}\NormalTok{(data}\SpecialCharTok{$}\NormalTok{gestation)}
\FunctionTok{print}\NormalTok{(gestation\_description)}
\end{Highlighting}
\end{Shaded}

\begin{verbatim}
##    Min. 1st Qu.  Median    Mean 3rd Qu.    Max. 
##   148.0   272.0   280.0   286.9   288.0   999.0
\end{verbatim}

\begin{Shaded}
\begin{Highlighting}[]
\NormalTok{parity\_description }\OtherTok{\textless{}{-}} \FunctionTok{summary}\NormalTok{(data}\SpecialCharTok{$}\NormalTok{parity)}
\FunctionTok{print}\NormalTok{(parity\_description)}
\end{Highlighting}
\end{Shaded}

\begin{verbatim}
##    Min. 1st Qu.  Median    Mean 3rd Qu.    Max. 
##  0.0000  0.0000  0.0000  0.2549  1.0000  1.0000
\end{verbatim}

\begin{Shaded}
\begin{Highlighting}[]
\NormalTok{age\_description }\OtherTok{\textless{}{-}} \FunctionTok{summary}\NormalTok{(data}\SpecialCharTok{$}\NormalTok{age)}
\FunctionTok{print}\NormalTok{(age\_description)}
\end{Highlighting}
\end{Shaded}

\begin{verbatim}
##    Min. 1st Qu.  Median    Mean 3rd Qu.    Max. 
##   15.00   23.00   26.00   27.37   31.00   99.00
\end{verbatim}

\begin{Shaded}
\begin{Highlighting}[]
\NormalTok{height\_description }\OtherTok{\textless{}{-}} \FunctionTok{summary}\NormalTok{(data}\SpecialCharTok{$}\NormalTok{height)}
\FunctionTok{print}\NormalTok{(height\_description)}
\end{Highlighting}
\end{Shaded}

\begin{verbatim}
##    Min. 1st Qu.  Median    Mean 3rd Qu.    Max. 
##   53.00   62.00   64.00   64.67   66.00   99.00
\end{verbatim}

\begin{Shaded}
\begin{Highlighting}[]
\NormalTok{weight\_description }\OtherTok{\textless{}{-}} \FunctionTok{summary}\NormalTok{(data}\SpecialCharTok{$}\NormalTok{weight)}
\FunctionTok{print}\NormalTok{(weight\_description)}
\end{Highlighting}
\end{Shaded}

\begin{verbatim}
##    Min. 1st Qu.  Median    Mean 3rd Qu.    Max. 
##      87     115     126     154     140     999
\end{verbatim}

\begin{Shaded}
\begin{Highlighting}[]
\NormalTok{smoke\_description }\OtherTok{\textless{}{-}} \FunctionTok{summary}\NormalTok{(data}\SpecialCharTok{$}\NormalTok{smoke)}
\FunctionTok{print}\NormalTok{(smoke\_description)}
\end{Highlighting}
\end{Shaded}

\begin{verbatim}
##    Min. 1st Qu.  Median    Mean 3rd Qu.    Max. 
##  0.0000  0.0000  0.0000  0.4644  1.0000  9.0000
\end{verbatim}

\subsubsection{Analysis}\label{analysis}

\begin{Shaded}
\begin{Highlighting}[]
\CommentTok{\# Your R code for data summary}
\end{Highlighting}
\end{Shaded}

\subsubsection{Conclusions}\label{conclusions}

Your conclusions about data processing and summaries.

\subsection{Question 1}\label{question-1}

\subsubsection{Methods}\label{methods-1}

\begin{Shaded}
\begin{Highlighting}[]
\CommentTok{\# Your R code for methods related to Question 1}
\end{Highlighting}
\end{Shaded}

\subsubsection{Analysis}\label{analysis-1}

\begin{Shaded}
\begin{Highlighting}[]
\CommentTok{\# Your R code for analysis related to Question 1}
\end{Highlighting}
\end{Shaded}

\subsubsection{Conclusions}\label{conclusions-1}

Your conclusions for Question 1.

\subsection{Question 2}\label{question-2}

\subsubsection{Methods}\label{methods-2}

\begin{Shaded}
\begin{Highlighting}[]
\CommentTok{\# Your R code for methods related to Question 2}
\end{Highlighting}
\end{Shaded}

\subsubsection{Analysis}\label{analysis-2}

\begin{Shaded}
\begin{Highlighting}[]
\CommentTok{\# Your R code for analysis related to Question 2}
\end{Highlighting}
\end{Shaded}

\subsubsection{Conclusions}\label{conclusions-2}

Your conclusions for Question 2.

\subsection{Question 3}\label{question-3}

\subsubsection{Methods}\label{methods-3}

\begin{Shaded}
\begin{Highlighting}[]
\CommentTok{\# Your R code for methods related to Question 3}
\end{Highlighting}
\end{Shaded}

\subsubsection{Analysis}\label{analysis-3}

\begin{Shaded}
\begin{Highlighting}[]
\CommentTok{\# Your R code for analysis related to Question 3}
\end{Highlighting}
\end{Shaded}

\subsubsection{Conclusions}\label{conclusions-3}

Your conclusions for Question 3.

\subsection{Question 4}\label{question-4}

\subsubsection{Methods}\label{methods-4}

\begin{Shaded}
\begin{Highlighting}[]
\CommentTok{\# Your R code for methods related to Question 4}
\end{Highlighting}
\end{Shaded}

\subsubsection{Analysis}\label{analysis-4}

\begin{Shaded}
\begin{Highlighting}[]
\CommentTok{\# Your R code for analysis related to Question 4}
\end{Highlighting}
\end{Shaded}

\subsubsection{Conclusions}\label{conclusions-4}

Your conclusions for Question 4.

\section{Advanced Analysis}\label{advanced-analysis}

\subsection{Additional Research
Question}\label{additional-research-question}

\subsubsection{Methods}\label{methods-5}

\begin{Shaded}
\begin{Highlighting}[]
\CommentTok{\# Your R code for methods related to the advanced question}
\end{Highlighting}
\end{Shaded}

\subsubsection{Analysis}\label{analysis-5}

\begin{Shaded}
\begin{Highlighting}[]
\CommentTok{\# Your R code for analysis related to the advanced question}
\end{Highlighting}
\end{Shaded}

\subsubsection{Conclusions}\label{conclusions-5}

Your conclusions for the advanced analysis.

\section{Conclusions and Discussion}\label{conclusions-and-discussion}

\subsection{Summary of Findings}\label{summary-of-findings}

Reprise the questions and goals of the analysis stated in the
introduction. Summarize the findings and compare them to the original
goals.

\subsection{Discussion}\label{discussion}

Additional observations or details gleaned from the analysis section.
Discuss relevance to the science and other studies, if applicable.
Address data limitations. Raise new questions and suggest future work.

\section{Appendix}\label{appendix}

Include any additional technical details, tables, or figures that
support your analysis but would disrupt the flow of the main text if
included earlier.

\begin{Shaded}
\begin{Highlighting}[]
\CommentTok{\# Additional R code or output can be included here}
\end{Highlighting}
\end{Shaded}


\end{document}
