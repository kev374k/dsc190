% Options for packages loaded elsewhere
\PassOptionsToPackage{unicode}{hyperref}
\PassOptionsToPackage{hyphens}{url}
%
\documentclass[
]{article}
\usepackage{amsmath,amssymb}
\usepackage{iftex}
\ifPDFTeX
  \usepackage[T1]{fontenc}
  \usepackage[utf8]{inputenc}
  \usepackage{textcomp} % provide euro and other symbols
\else % if luatex or xetex
  \usepackage{unicode-math} % this also loads fontspec
  \defaultfontfeatures{Scale=MatchLowercase}
  \defaultfontfeatures[\rmfamily]{Ligatures=TeX,Scale=1}
\fi
\usepackage{lmodern}
\ifPDFTeX\else
  % xetex/luatex font selection
\fi
% Use upquote if available, for straight quotes in verbatim environments
\IfFileExists{upquote.sty}{\usepackage{upquote}}{}
\IfFileExists{microtype.sty}{% use microtype if available
  \usepackage[]{microtype}
  \UseMicrotypeSet[protrusion]{basicmath} % disable protrusion for tt fonts
}{}
\makeatletter
\@ifundefined{KOMAClassName}{% if non-KOMA class
  \IfFileExists{parskip.sty}{%
    \usepackage{parskip}
  }{% else
    \setlength{\parindent}{0pt}
    \setlength{\parskip}{6pt plus 2pt minus 1pt}}
}{% if KOMA class
  \KOMAoptions{parskip=half}}
\makeatother
\usepackage{xcolor}
\usepackage[margin=1in]{geometry}
\usepackage{color}
\usepackage{fancyvrb}
\newcommand{\VerbBar}{|}
\newcommand{\VERB}{\Verb[commandchars=\\\{\}]}
\DefineVerbatimEnvironment{Highlighting}{Verbatim}{commandchars=\\\{\}}
% Add ',fontsize=\small' for more characters per line
\usepackage{framed}
\definecolor{shadecolor}{RGB}{248,248,248}
\newenvironment{Shaded}{\begin{snugshade}}{\end{snugshade}}
\newcommand{\AlertTok}[1]{\textcolor[rgb]{0.94,0.16,0.16}{#1}}
\newcommand{\AnnotationTok}[1]{\textcolor[rgb]{0.56,0.35,0.01}{\textbf{\textit{#1}}}}
\newcommand{\AttributeTok}[1]{\textcolor[rgb]{0.13,0.29,0.53}{#1}}
\newcommand{\BaseNTok}[1]{\textcolor[rgb]{0.00,0.00,0.81}{#1}}
\newcommand{\BuiltInTok}[1]{#1}
\newcommand{\CharTok}[1]{\textcolor[rgb]{0.31,0.60,0.02}{#1}}
\newcommand{\CommentTok}[1]{\textcolor[rgb]{0.56,0.35,0.01}{\textit{#1}}}
\newcommand{\CommentVarTok}[1]{\textcolor[rgb]{0.56,0.35,0.01}{\textbf{\textit{#1}}}}
\newcommand{\ConstantTok}[1]{\textcolor[rgb]{0.56,0.35,0.01}{#1}}
\newcommand{\ControlFlowTok}[1]{\textcolor[rgb]{0.13,0.29,0.53}{\textbf{#1}}}
\newcommand{\DataTypeTok}[1]{\textcolor[rgb]{0.13,0.29,0.53}{#1}}
\newcommand{\DecValTok}[1]{\textcolor[rgb]{0.00,0.00,0.81}{#1}}
\newcommand{\DocumentationTok}[1]{\textcolor[rgb]{0.56,0.35,0.01}{\textbf{\textit{#1}}}}
\newcommand{\ErrorTok}[1]{\textcolor[rgb]{0.64,0.00,0.00}{\textbf{#1}}}
\newcommand{\ExtensionTok}[1]{#1}
\newcommand{\FloatTok}[1]{\textcolor[rgb]{0.00,0.00,0.81}{#1}}
\newcommand{\FunctionTok}[1]{\textcolor[rgb]{0.13,0.29,0.53}{\textbf{#1}}}
\newcommand{\ImportTok}[1]{#1}
\newcommand{\InformationTok}[1]{\textcolor[rgb]{0.56,0.35,0.01}{\textbf{\textit{#1}}}}
\newcommand{\KeywordTok}[1]{\textcolor[rgb]{0.13,0.29,0.53}{\textbf{#1}}}
\newcommand{\NormalTok}[1]{#1}
\newcommand{\OperatorTok}[1]{\textcolor[rgb]{0.81,0.36,0.00}{\textbf{#1}}}
\newcommand{\OtherTok}[1]{\textcolor[rgb]{0.56,0.35,0.01}{#1}}
\newcommand{\PreprocessorTok}[1]{\textcolor[rgb]{0.56,0.35,0.01}{\textit{#1}}}
\newcommand{\RegionMarkerTok}[1]{#1}
\newcommand{\SpecialCharTok}[1]{\textcolor[rgb]{0.81,0.36,0.00}{\textbf{#1}}}
\newcommand{\SpecialStringTok}[1]{\textcolor[rgb]{0.31,0.60,0.02}{#1}}
\newcommand{\StringTok}[1]{\textcolor[rgb]{0.31,0.60,0.02}{#1}}
\newcommand{\VariableTok}[1]{\textcolor[rgb]{0.00,0.00,0.00}{#1}}
\newcommand{\VerbatimStringTok}[1]{\textcolor[rgb]{0.31,0.60,0.02}{#1}}
\newcommand{\WarningTok}[1]{\textcolor[rgb]{0.56,0.35,0.01}{\textbf{\textit{#1}}}}
\usepackage{graphicx}
\makeatletter
\newsavebox\pandoc@box
\newcommand*\pandocbounded[1]{% scales image to fit in text height/width
  \sbox\pandoc@box{#1}%
  \Gscale@div\@tempa{\textheight}{\dimexpr\ht\pandoc@box+\dp\pandoc@box\relax}%
  \Gscale@div\@tempb{\linewidth}{\wd\pandoc@box}%
  \ifdim\@tempb\p@<\@tempa\p@\let\@tempa\@tempb\fi% select the smaller of both
  \ifdim\@tempa\p@<\p@\scalebox{\@tempa}{\usebox\pandoc@box}%
  \else\usebox{\pandoc@box}%
  \fi%
}
% Set default figure placement to htbp
\def\fps@figure{htbp}
\makeatother
\setlength{\emergencystretch}{3em} % prevent overfull lines
\providecommand{\tightlist}{%
  \setlength{\itemsep}{0pt}\setlength{\parskip}{0pt}}
\setcounter{secnumdepth}{-\maxdimen} % remove section numbering
\usepackage{bookmark}
\IfFileExists{xurl.sty}{\usepackage{xurl}}{} % add URL line breaks if available
\urlstyle{same}
\hypersetup{
  pdftitle={Homework \#1},
  pdfauthor={Kevin Wong},
  hidelinks,
  pdfcreator={LaTeX via pandoc}}

\title{Homework \#1}
\author{Kevin Wong}
\date{2024-10-07}

\begin{document}
\maketitle

Data Cleaning

\begin{verbatim}
##   time like where freq busy educ sex age home math work own cdrom email grade
## 1  2.0    3     3    2    0    1   0  19    1    0   10   1     0     1     4
## 2  0.0    3     3    3    0    0   0  18    1    1    0   1     1     1     2
## 3  0.0    3     1    3    0    0   1  19    1    0    0   1     0     1     3
## 4  0.5    3     3    3    0    1   0  19    1    0    0   1     0     1     3
## 5  0.0    3     3    4    0    1   0  19    1    1    0   0     0     1     3
\end{verbatim}

\begin{verbatim}
##   action adv sim sport strategy relax coord challenge master bored other
## 1      0   0   0     0        1     1     0         1      1     0      
## 2      0   1   0     0        1     0     0         0      0     1      
## 3      1   0   0     1        1     1     0         0      0     0      
## 4      0   0   0     0        1     0     0         1      0     0      
## 5      0   0   0     0        1     1     0         1      1     0      
##   graphic time frust lonely rules cost boring friends point other2
## 1       0    1     0      0     0    1      0       0     1       
## 2       0    1     1      0     0    0      0       0     0       
## 3       0    0     0      0     0    1      0       0     0       
## 4       0    1     0      0     0    0      0       0     0       
## 5       0    0     0      0     1    1      0       0     0
\end{verbatim}

\begin{verbatim}
## 'data.frame':    91 obs. of  15 variables:
##  $ time : num  2 0 0 0.5 0 0 0 0 2 0 ...
##  $ like : int  3 3 3 3 3 3 4 3 3 3 ...
##  $ where: int  3 3 1 3 3 2 3 3 2 3 ...
##  $ freq : int  2 3 3 3 4 4 4 4 1 4 ...
##  $ busy : int  0 0 0 0 0 0 0 0 1 0 ...
##  $ educ : int  1 0 0 1 1 0 0 0 1 1 ...
##  $ sex  : int  0 0 1 0 0 1 1 0 1 1 ...
##  $ age  : int  19 18 19 19 19 19 20 19 19 19 ...
##  $ home : int  1 1 1 1 1 0 1 1 0 1 ...
##  $ math : int  0 1 0 0 1 0 1 0 0 1 ...
##  $ work : int  10 0 0 0 0 12 10 13 0 0 ...
##  $ own  : int  1 1 1 1 0 0 1 0 0 1 ...
##  $ cdrom: int  0 1 0 0 0 0 0 0 0 0 ...
##  $ email: int  1 1 1 1 1 0 1 1 0 1 ...
##  $ grade: int  4 2 3 3 3 3 3 3 4 4 ...
\end{verbatim}

\begin{verbatim}
## 'data.frame':    91 obs. of  21 variables:
##  $ action   : int  0 0 1 0 0 1 1 0 1 1 ...
##  $ adv      : int  0 1 0 0 0 0 0 0 0 1 ...
##  $ sim      : int  0 0 0 0 0 0 0 0 0 1 ...
##  $ sport    : int  0 0 1 0 0 1 1 1 1 0 ...
##  $ strategy : int  1 1 1 1 1 1 0 1 0 1 ...
##  $ relax    : int  1 0 1 0 1 1 1 1 0 1 ...
##  $ coord    : int  0 0 0 0 0 0 0 0 0 0 ...
##  $ challenge: int  1 0 0 1 1 0 0 0 0 0 ...
##  $ master   : int  1 0 0 0 1 1 0 0 0 0 ...
##  $ bored    : int  0 1 0 0 0 1 0 1 0 0 ...
##  $ other    : chr  " " " " " " " " ...
##  $ graphic  : int  0 0 0 0 0 0 0 1 1 0 ...
##  $ time     : int  1 1 0 1 0 1 0 0 0 0 ...
##  $ frust    : int  0 1 0 0 0 1 0 0 0 0 ...
##  $ lonely   : int  0 0 0 0 0 0 0 0 0 0 ...
##  $ rules    : int  0 0 0 0 1 0 1 0 0 1 ...
##  $ cost     : int  1 0 1 0 1 1 0 0 1 0 ...
##  $ boring   : int  0 0 0 0 0 0 0 0 1 0 ...
##  $ friends  : int  0 0 0 0 0 0 0 0 0 0 ...
##  $ point    : int  1 0 0 0 0 0 0 0 0 0 ...
##  $ other2   : chr  " " " " " " " " ...
\end{verbatim}

Through this, we notice that the all of the columns in our first dataset
are numbers/integers, which is correct. However, for our second dataset,
our variables are mostly ints which make sense, but there are two extra
columns, `other' and `other2' are present but are not in the second part
of the survey. This means that there might be a discrepancy in the data,
because it should be numeric, even though it isn't. Let's create a
function to detect non-numeric columns, and show us them.

\begin{Shaded}
\begin{Highlighting}[]
\NormalTok{survey\_p2 }\OtherTok{\textless{}{-}}\NormalTok{ survey\_p2[, }\SpecialCharTok{!}\NormalTok{(}\FunctionTok{names}\NormalTok{(survey\_p2) }\SpecialCharTok{\%in\%} \FunctionTok{c}\NormalTok{(}\StringTok{"other"}\NormalTok{, }\StringTok{"other2"}\NormalTok{))]}
\FunctionTok{all}\NormalTok{(}\FunctionTok{sapply}\NormalTok{(survey\_p2, is.numeric))}
\end{Highlighting}
\end{Shaded}

\begin{verbatim}
## [1] TRUE
\end{verbatim}

\begin{Shaded}
\begin{Highlighting}[]
\NormalTok{rows\_with\_all\_zeros }\OtherTok{\textless{}{-}} \FunctionTok{apply}\NormalTok{(survey\_p2, }\DecValTok{1}\NormalTok{, }\ControlFlowTok{function}\NormalTok{(row) }\FunctionTok{all}\NormalTok{(row }\SpecialCharTok{==} \DecValTok{0}\NormalTok{))}
\NormalTok{survey\_p2[rows\_with\_all\_zeros, ]}
\end{Highlighting}
\end{Shaded}

\begin{verbatim}
##      action adv sim sport strategy relax coord challenge master bored graphic
## NA       NA  NA  NA    NA       NA    NA    NA        NA     NA    NA      NA
## NA.1     NA  NA  NA    NA       NA    NA    NA        NA     NA    NA      NA
## NA.2     NA  NA  NA    NA       NA    NA    NA        NA     NA    NA      NA
## NA.3     NA  NA  NA    NA       NA    NA    NA        NA     NA    NA      NA
##      time frust lonely rules cost boring friends point
## NA     NA    NA     NA    NA   NA     NA      NA    NA
## NA.1   NA    NA     NA    NA   NA     NA      NA    NA
## NA.2   NA    NA     NA    NA   NA     NA      NA    NA
## NA.3   NA    NA     NA    NA   NA     NA      NA    NA
\end{verbatim}

In order to further clean the data, we are going to combine the tables
together and merge them in order to create one large table. For this, we
are going to go with the assumption that each row in both tables
represents the same person, since each student was assigned a unique
number.

\begin{verbatim}
##   time like where freq busy educ sex age home math work own cdrom email grade
## 1  2.0    3     3    2    0    1   0  19    1    0   10   1     0     1     4
## 2  0.0    3     3    3    0    0   0  18    1    1    0   1     1     1     2
## 3  0.0    3     1    3    0    0   1  19    1    0    0   1     0     1     3
## 4  0.5    3     3    3    0    1   0  19    1    0    0   1     0     1     3
## 5  0.0    3     3    4    0    1   0  19    1    1    0   0     0     1     3
##   action adv sim sport strategy relax coord challenge master bored graphic time
## 1      0   0   0     0        1     1     0         1      1     0       0    1
## 2      0   1   0     0        1     0     0         0      0     1       0    1
## 3      1   0   0     1        1     1     0         0      0     0       0    0
## 4      0   0   0     0        1     0     0         1      0     0       0    1
## 5      0   0   0     0        1     1     0         1      1     0       0    0
##   frust lonely rules cost boring friends point
## 1     0      0     0    1      0       0     1
## 2     1      0     0    0      0       0     0
## 3     0      0     0    1      0       0     0
## 4     0      0     0    0      0       0     0
## 5     0      0     1    1      0       0     0
\end{verbatim}

Below is a table showing all rows that contain at least one NA value. As
we can see, NA values only exist in the variables from the part 2
survey.

\begin{verbatim}
##    time like where freq busy educ sex age home math work own cdrom email grade
## 53  0.5    3     2    2    0    0   1  19    1    0   16   1     0     1     3
## 54  3.0    2     3    1    0    1   1  18    1    0    7   1     0     1     3
## 55  0.0    3     1    3    0    0   1  19    0    0   15   0     0     1     3
## 56  0.0    4     3    3    0    1   0  21    1    0    5   1     0     1     4
##    action adv sim sport strategy relax coord challenge master bored graphic
## 53     NA  NA  NA    NA       NA    NA    NA        NA     NA    NA      NA
## 54     NA  NA  NA    NA       NA    NA    NA        NA     NA    NA      NA
## 55     NA  NA  NA    NA       NA    NA    NA        NA     NA    NA      NA
## 56     NA  NA  NA    NA       NA    NA    NA        NA     NA    NA      NA
##    time frust lonely rules cost boring friends point
## 53   NA    NA     NA    NA   NA     NA      NA    NA
## 54   NA    NA     NA    NA   NA     NA      NA    NA
## 55   NA    NA     NA    NA   NA     NA      NA    NA
## 56   NA    NA     NA    NA   NA     NA      NA    NA
\end{verbatim}

Now, let's get rid of rows that contain NA values, just to get rid of
inconsistent data.

\begin{verbatim}
## [1] 87 34
\end{verbatim}

Looking through this, we had 4 rows that contained NA values, so now we
only have 87 rows left.

Notes: 1) busy, educ, sex, home, math, own, cdrom, email, action, adv,
sim, sport, strategy, graphic, relax, coord, challenge, bored, time,
frust, lonely, rules, cost, boring, friends, point are binary, so their
distributions will be concentrated at 0 and 1 2) time, age, and work are
continuous variables, may be similar to a normal distribution 3) like,
freq, and grade are categorical, ordinal variables 4) where is a nominal
variable 5) 99 shouldn't be included in the data

\begin{Shaded}
\begin{Highlighting}[]
\ControlFlowTok{for}\NormalTok{ (col }\ControlFlowTok{in} \FunctionTok{names}\NormalTok{(cleaned\_df)) \{}
\NormalTok{    valid\_data }\OtherTok{\textless{}{-}}\NormalTok{ cleaned\_df[[col]][cleaned\_df[[col]] }\SpecialCharTok{!=} \DecValTok{99}\NormalTok{]}

    \ControlFlowTok{if}\NormalTok{ (}\FunctionTok{length}\NormalTok{(valid\_data) }\SpecialCharTok{\textgreater{}} \DecValTok{0}\NormalTok{) \{}
        \FunctionTok{hist}\NormalTok{(valid\_data, }\AttributeTok{main =} \FunctionTok{paste}\NormalTok{(}\StringTok{"Histogram of"}\NormalTok{, col), }\AttributeTok{xlab =}\NormalTok{ col, }\AttributeTok{breaks=}\DecValTok{30}\NormalTok{)}
\NormalTok{    \} }\ControlFlowTok{else}\NormalTok{ \{}
        \FunctionTok{message}\NormalTok{(}\FunctionTok{paste}\NormalTok{(}\StringTok{"No valid data for column"}\NormalTok{, col))}
\NormalTok{    \}}
\NormalTok{\}}
\end{Highlighting}
\end{Shaded}

\pandocbounded{\includegraphics[keepaspectratio]{hw2_files/figure-latex/unnamed-chunk-9-1.pdf}}
\pandocbounded{\includegraphics[keepaspectratio]{hw2_files/figure-latex/unnamed-chunk-9-2.pdf}}
\pandocbounded{\includegraphics[keepaspectratio]{hw2_files/figure-latex/unnamed-chunk-9-3.pdf}}
\pandocbounded{\includegraphics[keepaspectratio]{hw2_files/figure-latex/unnamed-chunk-9-4.pdf}}
\pandocbounded{\includegraphics[keepaspectratio]{hw2_files/figure-latex/unnamed-chunk-9-5.pdf}}
\pandocbounded{\includegraphics[keepaspectratio]{hw2_files/figure-latex/unnamed-chunk-9-6.pdf}}
\pandocbounded{\includegraphics[keepaspectratio]{hw2_files/figure-latex/unnamed-chunk-9-7.pdf}}
\pandocbounded{\includegraphics[keepaspectratio]{hw2_files/figure-latex/unnamed-chunk-9-8.pdf}}
\pandocbounded{\includegraphics[keepaspectratio]{hw2_files/figure-latex/unnamed-chunk-9-9.pdf}}
\pandocbounded{\includegraphics[keepaspectratio]{hw2_files/figure-latex/unnamed-chunk-9-10.pdf}}
\pandocbounded{\includegraphics[keepaspectratio]{hw2_files/figure-latex/unnamed-chunk-9-11.pdf}}
\pandocbounded{\includegraphics[keepaspectratio]{hw2_files/figure-latex/unnamed-chunk-9-12.pdf}}
\pandocbounded{\includegraphics[keepaspectratio]{hw2_files/figure-latex/unnamed-chunk-9-13.pdf}}
\pandocbounded{\includegraphics[keepaspectratio]{hw2_files/figure-latex/unnamed-chunk-9-14.pdf}}
\pandocbounded{\includegraphics[keepaspectratio]{hw2_files/figure-latex/unnamed-chunk-9-15.pdf}}
\pandocbounded{\includegraphics[keepaspectratio]{hw2_files/figure-latex/unnamed-chunk-9-16.pdf}}
\pandocbounded{\includegraphics[keepaspectratio]{hw2_files/figure-latex/unnamed-chunk-9-17.pdf}}
\pandocbounded{\includegraphics[keepaspectratio]{hw2_files/figure-latex/unnamed-chunk-9-18.pdf}}
\pandocbounded{\includegraphics[keepaspectratio]{hw2_files/figure-latex/unnamed-chunk-9-19.pdf}}
\pandocbounded{\includegraphics[keepaspectratio]{hw2_files/figure-latex/unnamed-chunk-9-20.pdf}}
\pandocbounded{\includegraphics[keepaspectratio]{hw2_files/figure-latex/unnamed-chunk-9-21.pdf}}
\pandocbounded{\includegraphics[keepaspectratio]{hw2_files/figure-latex/unnamed-chunk-9-22.pdf}}
\pandocbounded{\includegraphics[keepaspectratio]{hw2_files/figure-latex/unnamed-chunk-9-23.pdf}}
\pandocbounded{\includegraphics[keepaspectratio]{hw2_files/figure-latex/unnamed-chunk-9-24.pdf}}
\pandocbounded{\includegraphics[keepaspectratio]{hw2_files/figure-latex/unnamed-chunk-9-25.pdf}}
\pandocbounded{\includegraphics[keepaspectratio]{hw2_files/figure-latex/unnamed-chunk-9-26.pdf}}
\pandocbounded{\includegraphics[keepaspectratio]{hw2_files/figure-latex/unnamed-chunk-9-27.pdf}}
\pandocbounded{\includegraphics[keepaspectratio]{hw2_files/figure-latex/unnamed-chunk-9-28.pdf}}
\pandocbounded{\includegraphics[keepaspectratio]{hw2_files/figure-latex/unnamed-chunk-9-29.pdf}}
\pandocbounded{\includegraphics[keepaspectratio]{hw2_files/figure-latex/unnamed-chunk-9-30.pdf}}
\pandocbounded{\includegraphics[keepaspectratio]{hw2_files/figure-latex/unnamed-chunk-9-31.pdf}}
\pandocbounded{\includegraphics[keepaspectratio]{hw2_files/figure-latex/unnamed-chunk-9-32.pdf}}
\pandocbounded{\includegraphics[keepaspectratio]{hw2_files/figure-latex/unnamed-chunk-9-33.pdf}}
\pandocbounded{\includegraphics[keepaspectratio]{hw2_files/figure-latex/unnamed-chunk-9-34.pdf}}

Looking through the histograms, there are multiple values that are coded
as 99 because students left some questions unaswered or improperly
answered. This means that when we filter the data, we need to make sure
that the data that we get don't include the non-responses

\section{1}\label{section}

In order to provide a solid estimation of the amount of people who
played a video game in the week prior to the survey, let's filter the
amount of people who had over 0 hours last week as well as get rid of
the people who didn't respond to the prompt (99)

\begin{verbatim}
## [1] "Point Estimate of the fraction of students who played a video game in the week prior to the survey:  0.3678"
\end{verbatim}

Now, let's use a 95\% confidence interval to construct an interval
estimate for this proportion

\begin{verbatim}
## [1] "95% Confidence Interval: [0.2665, 0.4691]"
\end{verbatim}

The distinction between the point estimate and the interval estimate is
that they serve different purposes in statistical inference. For
instance, a point estimate is a value that provides the best
guess/estimate for a population parameter. On the other hand, an
interval estimate is an estimate that provides a range of values where
the true population parameter is expected to lie in, and specifically in
this case, with a 95\% confidence. Point Estimates are simple and
provide a direct measurement, while intervals are more uncertain, but
are more robust as data scales up.

\section{2}\label{section-1}

First, in order to determine the amount of time someone spent playing
video games, we should look at the `time' column as well as the `freq'
column so we can compare the two values. In order to determine

\begin{verbatim}
##   frequency   avg_time median_time count
## 1         1 4.62500000           2     8
## 2         2 2.61481481           2    27
## 3         3 0.06250000           0    16
## 4         4 0.04347826           0    23
\end{verbatim}

\pandocbounded{\includegraphics[keepaspectratio]{hw2_files/figure-latex/unnamed-chunk-12-1.pdf}}

This time let's filter for exam week. One assumption that we made
regarding this is that students were more likely to study instead of
game for the week, so for the next visualization, let's ignore students
who reported 0 hours of gaming in the previous week.

\begin{verbatim}
##   frequency avg_time median_time count
## 1         1 6.166667         3.0     6
## 2         2 3.069565         2.0    23
## 3         3 0.500000         0.5     2
## 4         4 1.000000         1.0     1
\end{verbatim}

\pandocbounded{\includegraphics[keepaspectratio]{hw2_files/figure-latex/unnamed-chunk-13-1.pdf}}

Looking through these two boxplots and their data summaries, there is an
obvious difference, especially for students with a daily or weekly
frequency of play. For instance, our adjusted comparison shows that
those who played (`freq' = 1) daily during an exam week typically had a
lower amount of hours played, with a 1.5 hour difference between our
exam week data and our filtered data. Similarly, our boxplot also shows
a difference in the interquartile range between the two datasets for
those who played daily, as during a normal week, students were more
likely to play more and had a wider variety of playtimes, which can be
inferred from the two box plots. A similar comparison can be made from
students who played weekly (`freq' = 2). There's a \textasciitilde0.4
hour increase in normal weeks versus the exam week. Another interesting
observation is that the adjusted data seems to have a smaller
interquartile range compared to it's original exam week data, somewhat
implying that those who play weekly usually play a more set amount of
hours that usually doesn't diverge. Frequencies 3 and 4 don't have a
large difference, just due to the nature of their descriptions. If
someone only plays games monthly or semesterly, then they are less
likely to have major differences during an exam week versus a normal
week. Overall, the fact that there was an exam in the week prior to the
survey tells us this:

\begin{itemize}
\tightlist
\item
  On average, students played less hours than they normally would on a
  week without an exam
\item
  Those who reported that they play daily or weekly typically have the
  largest difference between their average hours played during an exam
  versus a normal week
\item
  Those who reported that they play monthly or semesterly have small or
  negligible differences between their average hours played due to the
  nature and scarcity of their data.
\end{itemize}

\section{3}\label{section-2}

The point estimate for the average amount of time spent playing video
games in the week prior to the survey is simply the sample mean. We will
also calculate the standard error (SE) to represent the expected
variability of our sample mean from the true population mean of time
spent playing video games. This SE will also be used to construct our
95\% confidence interval. The SE formula must use the \textbf{population
correction factor} because we are sampling \textbf{without replacement}
from a relatively small population of 314.
\[SE = {σ \over \sqrt{n}} \times \sqrt{N-n \over N - 1}\]

\begin{Shaded}
\begin{Highlighting}[]
\NormalTok{point\_estimate\_average }\OtherTok{\textless{}{-}} \FunctionTok{mean}\NormalTok{(survey\_p1}\SpecialCharTok{$}\NormalTok{time, }\AttributeTok{na.rm=}\ConstantTok{TRUE}\NormalTok{)}
\NormalTok{N }\OtherTok{\textless{}{-}} \DecValTok{314}
\NormalTok{n }\OtherTok{\textless{}{-}} \FunctionTok{length}\NormalTok{(survey\_p1}\SpecialCharTok{$}\NormalTok{time)}

\NormalTok{sample\_sd }\OtherTok{\textless{}{-}} \FunctionTok{sd}\NormalTok{(survey\_p1}\SpecialCharTok{$}\NormalTok{time, }\AttributeTok{na.rm =} \ConstantTok{TRUE}\NormalTok{)}
\NormalTok{pop\_correction\_factor }\OtherTok{=} \FunctionTok{sqrt}\NormalTok{((N }\SpecialCharTok{{-}}\NormalTok{ n) }\SpecialCharTok{/}\NormalTok{ (N }\SpecialCharTok{{-}} \DecValTok{1}\NormalTok{))}
\NormalTok{standard\_error }\OtherTok{\textless{}{-}}\NormalTok{ (sample\_sd }\SpecialCharTok{/} \FunctionTok{sqrt}\NormalTok{(n)) }\SpecialCharTok{*}\NormalTok{ pop\_correction\_factor}

\FunctionTok{print}\NormalTok{(}\FunctionTok{paste}\NormalTok{(}\StringTok{"point\_estimate\_average:"}\NormalTok{, point\_estimate\_average))}
\end{Highlighting}
\end{Shaded}

\begin{verbatim}
## [1] "point_estimate_average: 1.24285714285714"
\end{verbatim}

\begin{Shaded}
\begin{Highlighting}[]
\FunctionTok{print}\NormalTok{(}\FunctionTok{paste}\NormalTok{(}\StringTok{"Standard Error:"}\NormalTok{, standard\_error))}
\end{Highlighting}
\end{Shaded}

\begin{verbatim}
## [1] "Standard Error: 0.334203627625554"
\end{verbatim}

Construct the 95\% Confidence Interval.

\begin{Shaded}
\begin{Highlighting}[]
\NormalTok{alpha }\OtherTok{\textless{}{-}} \FloatTok{0.05}
\NormalTok{z\_value }\OtherTok{\textless{}{-}} \FunctionTok{qnorm}\NormalTok{(}\DecValTok{1} \SpecialCharTok{{-}}\NormalTok{ alpha }\SpecialCharTok{/} \DecValTok{2}\NormalTok{)}

\NormalTok{lower\_interval\_estimate\_average }\OtherTok{\textless{}{-}}\NormalTok{ point\_estimate\_average }\SpecialCharTok{{-}}\NormalTok{ z\_value }\SpecialCharTok{*}\NormalTok{ standard\_error}
\NormalTok{upper\_interval\_estimate\_average }\OtherTok{\textless{}{-}}\NormalTok{ point\_estimate\_average }\SpecialCharTok{+}\NormalTok{ z\_value }\SpecialCharTok{*}\NormalTok{ standard\_error}

\FunctionTok{print}\NormalTok{(}\FunctionTok{paste}\NormalTok{(}\StringTok{"lower\_interval\_estimate\_average:"}\NormalTok{, lower\_interval\_estimate\_average))}
\end{Highlighting}
\end{Shaded}

\begin{verbatim}
## [1] "lower_interval_estimate_average: 0.587830069208421"
\end{verbatim}

\begin{Shaded}
\begin{Highlighting}[]
\FunctionTok{print}\NormalTok{(}\FunctionTok{paste}\NormalTok{(}\StringTok{"upper\_interval\_estimate\_average:"}\NormalTok{, upper\_interval\_estimate\_average))}
\end{Highlighting}
\end{Shaded}

\begin{verbatim}
## [1] "upper_interval_estimate_average: 1.89788421650587"
\end{verbatim}

Next, we perform a bootstrap simulation to estimate the sampling
distribution of average time spent playing games. If this estimated
sampling distribution is approximately normal, then our confidence
interval is appropriate, but if it's not normal, then the approach taken
is not appropriate.

\begin{Shaded}
\begin{Highlighting}[]
\FunctionTok{library}\NormalTok{(boot)}

\CommentTok{\# Calculate the sample mean for each bootstrapped sample}
\NormalTok{bootstrap\_mean }\OtherTok{\textless{}{-}} \ControlFlowTok{function}\NormalTok{(data, indices) \{}
  \FunctionTok{return}\NormalTok{(}\FunctionTok{mean}\NormalTok{(data[indices], }\AttributeTok{na.rm =} \ConstantTok{TRUE}\NormalTok{))}
\NormalTok{\}}

\CommentTok{\# Perform bootstrapping with 1000 iterations}
\NormalTok{bootstrap\_results }\OtherTok{\textless{}{-}} \FunctionTok{boot}\NormalTok{(survey\_p1}\SpecialCharTok{$}\NormalTok{time, bootstrap\_mean, }\AttributeTok{R =} \DecValTok{1000}\NormalTok{)}
\NormalTok{bootstrap\_means }\OtherTok{\textless{}{-}}\NormalTok{ bootstrap\_results}\SpecialCharTok{$}\NormalTok{t}
\end{Highlighting}
\end{Shaded}

Now, I construct another 95\% confidence interval using the percentiles
of my bootstrapped sample means and compare it to the earlier confidence
interval which was constructed using the point estimate and an
assumption of a normal distribution. If they are significantly
different, the interval estimate may not be appropriate for this
situation.

\begin{Shaded}
\begin{Highlighting}[]
\NormalTok{bootstrap\_ci }\OtherTok{\textless{}{-}} \FunctionTok{boot.ci}\NormalTok{(bootstrap\_results, }\AttributeTok{type =} \StringTok{"perc"}\NormalTok{)}

\CommentTok{\# Extract lower and upper bounds of the 95\% Bootstrap CI}
\NormalTok{lower\_bound\_bootstrap }\OtherTok{\textless{}{-}}\NormalTok{ bootstrap\_ci}\SpecialCharTok{$}\NormalTok{percent[}\DecValTok{4}\NormalTok{]}
\NormalTok{upper\_bound\_bootstrap }\OtherTok{\textless{}{-}}\NormalTok{ bootstrap\_ci}\SpecialCharTok{$}\NormalTok{percent[}\DecValTok{5}\NormalTok{]}

\FunctionTok{hist}\NormalTok{(bootstrap\_means, }\AttributeTok{main =} \StringTok{"Bootstrap Sample Means with 95\% CIs"}\NormalTok{, }
     \AttributeTok{xlab =} \StringTok{"Sample Mean Time Spent"}\NormalTok{, }
     \AttributeTok{col =} \StringTok{"lightblue"}\NormalTok{, }
     \AttributeTok{border =} \StringTok{"black"}\NormalTok{, }
     \AttributeTok{probability =} \ConstantTok{TRUE}\NormalTok{)}

\CommentTok{\# Add vertical lines for the confidence intervals}
\FunctionTok{abline}\NormalTok{(}\AttributeTok{v =}\NormalTok{ lower\_bound\_bootstrap, }\AttributeTok{col =} \StringTok{"red"}\NormalTok{, }\AttributeTok{lwd =} \DecValTok{2}\NormalTok{, }\AttributeTok{lty =} \DecValTok{2}\NormalTok{)  }\CommentTok{\# Lower CI bound}
\FunctionTok{abline}\NormalTok{(}\AttributeTok{v =}\NormalTok{ upper\_bound\_bootstrap, }\AttributeTok{col =} \StringTok{"red"}\NormalTok{, }\AttributeTok{lwd =} \DecValTok{2}\NormalTok{, }\AttributeTok{lty =} \DecValTok{2}\NormalTok{)  }\CommentTok{\# Upper CI bound}
\FunctionTok{abline}\NormalTok{(}\AttributeTok{v =}\NormalTok{ lower\_interval\_estimate\_average, }\AttributeTok{col =} \StringTok{"blue"}\NormalTok{, }\AttributeTok{lwd =} \DecValTok{2}\NormalTok{, }\AttributeTok{lty =} \DecValTok{1}\NormalTok{)}
\FunctionTok{abline}\NormalTok{(}\AttributeTok{v =}\NormalTok{ upper\_interval\_estimate\_average, }\AttributeTok{col =} \StringTok{"blue"}\NormalTok{, }\AttributeTok{lwd =} \DecValTok{2}\NormalTok{, }\AttributeTok{lty =} \DecValTok{1}\NormalTok{)}


\FunctionTok{legend}\NormalTok{(}\StringTok{"topright"}\NormalTok{, }\AttributeTok{legend =} \FunctionTok{c}\NormalTok{(}\StringTok{"Lower Bootstrap CI"}\NormalTok{, }\StringTok{"Upper Bootstrap CI"}\NormalTok{, }\StringTok{"Lower 95\% CI"}\NormalTok{, }\StringTok{"Upper 95\% CI"}\NormalTok{),}
       \AttributeTok{col =} \FunctionTok{c}\NormalTok{(}\StringTok{"red"}\NormalTok{, }\StringTok{"red"}\NormalTok{, }\StringTok{"blue"}\NormalTok{, }\StringTok{"blue"}\NormalTok{), }\AttributeTok{lty =} \FunctionTok{c}\NormalTok{(}\DecValTok{2}\NormalTok{, }\DecValTok{2}\NormalTok{, }\DecValTok{1}\NormalTok{, }\DecValTok{1}\NormalTok{), }\AttributeTok{lwd =} \DecValTok{2}\NormalTok{)}
\end{Highlighting}
\end{Shaded}

\pandocbounded{\includegraphics[keepaspectratio]{hw2_files/figure-latex/unnamed-chunk-17-1.pdf}}

\section{Advanced Analysis}\label{advanced-analysis}

Our advanced analysis focuses on the intersectionality between people
that like video games and which types of video games they prefer. By
doing this, we can see the most popular types of video games that these
students prefer, along with the average amount of time they play video
games a week, potentially showing us which games people spent the most
time on.

First, we filter the data. It's important to note that we ignore people
who skipped this question, because the instructions state that those who
disliked video games were instructed to do so.

\begin{verbatim}
## [1] 75 34
\end{verbatim}

Now, let's capture the most popular video game types from this data.

\pandocbounded{\includegraphics[keepaspectratio]{hw2_files/figure-latex/unnamed-chunk-19-1.pdf}}

Now, let's compare the amount of time people spent playing these games.

\pandocbounded{\includegraphics[keepaspectratio]{hw2_files/figure-latex/unnamed-chunk-20-1.pdf}}

Our two graphs, especially when put next to each other, show us another
interesting detail. Although strategy was the most popular game genre
chosen by people, with the action and sport genres falling slightly
behind, when we look through the total time spent playing each game,
action was the highest, with sport and then strategy falling behind.
This means that there's a decent chance that the people who play
strategy games don't typically play the game for a longer time, while
those who played action and sports games play for longer durations. On
another note, sim was the least popular genre of game in the survey, and
that pattern still shows in the total time spent. People who mentioned
they like simulation games spent very little time actually playing them,
leading us to believe that those who play these simulation games either
don't play them very often or play them for short periods of time
compared to their peers who played action or sport games.

Of course, this is not a conclusive study. Firstly, there's inherent
bias because this survey was taken during an exam week, likely leading
to students playing less than they normally would. Additionally, the way
we filtered our data may be potentially flawed, as students were allowed
to choose up to 3 of their favorite genres, not just one. This means
that our graph of total time spent on each category is layered and can
be duplicated, because if a student likes to play multiple genres of
games, they are stacked on top of their respective genres, potentially
leading to more visible and exaggerated graphs than what would normally
be. However, even with that discrepancy, we still believe that these two
graphs show us concrete differences in many of the genres and how
students viewed the genres differently, both in terms of popularity as
well as personal enjoyment.

\end{document}
